\documentclass[10pt,letterpaper]{article}

\usepackage{palatino,url}

\usepackage[left=2cm,top=3cm,bottom=3cm,right=2cm,nohead,nofoot]{geometry}

\title{Perceus and Caos NSA Quick-Start}

\author{www.Infiscale.com}

\date{\today}

\begin{document}

\maketitle

\addtolength{\parskip}{0.5\baselineskip}

\section{Introduction}
Historically Warewulf has been the de facto stateless cluster manager in
HPC. Perceus (the next generation of Warewulf) has taken cluster management
to the next level. It has already proved to be easier to use and administer,
more scalable, more flexible, and more supportable.

While Perceus can be (and is being) packaged on various distributions, it has
been incorporated with Caos NSA by the Perceus and Caos developers in a manner
that makes it even easier to implement and use!  This is the fastest way to
implement an HPC cluster and is the basis of extreme clustering of many
thousands of nodes.

The intent of this document is to provide users a fast, pain-free installation
and configuration of Caos NSA with Perceus to create a turnkey HPC cluster. For
more detailed information on general Perceus concepts and commands, the Perceus
User Guide should be read.

\subsection{About this document}
This document is designed to satisfy the general needs of the Perceus user.
The full manual, programming, building, and integration guides are available
from Infiscale.com as a licensed product. Licensing Perceus and/or these
documents from Infiscale.com supports the development of Infiscale's open
source projects.

This guide will describe the quick start procedure for setting up Perceus
on Caos NSA optionally using Intel's Rapid Boot.

The installation described here is also the basis of the only freely
available, Intel Cluster Ready certified solution.

\subsection{Perceus}
Perceus is the next generation cluster and enterprise tool kit for the
deployment, provisioning, and management of groups of servers. Employing the
power of the Perceus OS and framework, the user can quickly purpose a machine
out of the box. Perceus truly makes the computer a commodity, allowing an
organization to manage large quantities of machines in a scalable fashion.

Created by the developers of Warewulf, one of the most widely-utilized Linux
cluster tool-kits and the de facto standard open source stateless solution for
many years now, Perceus redefines the limits of scalability, flexibility, and
simplicity.  Perceus was architected from its inception to leverage the
knowledge and experience gained from the original Warewulf project while at the
same time addressing some of its perceived shortcomings.  This unique vantage
point enables us to deliver what we feel is the definitive solution for all
cluster and highly-scalable platforms regardless of their size or specific
requirements.

While other cluster toolkit solutions have traded ease of use for scalability,
flexibility, and/or simplicity, Perceus leverages a revolutionary new design
which makes it extremely well-suited to fit the needs of a wide array of
environments, from highly-scalable and performance-centric clustering to new
system provisioning. Without making any sacrifices, Perceus can be extremely
simple to use and "turnkey," or it can be used as a development-enabling
platform designed for very specialized roles.

\noindent Additional information, news, resources and downloads can be found
at:

{\ttfamily
\indent \url{http://www.perceus.org/}
}

\subsection{Caos NSA}
Caos NSA is the second generation operating system developed by
Infiscale.com. Created by the original developers of Perceus, Warewulf,
Centos, Caos, and GravityOS it is designed for high performance, simplicity
and scalability. Linux doing what it has always done best.

With a large emphasis on performance, Caos NSA is dedicated to providing a
proper solution for HPC. Focused on being light weight, streamlined, simple,
fast and clean, it also comes with integration with the Open Fabrics Alliance
InfiniBand stack, recent and security enabled kernels, Open MPI, integrated
scheduler, Intel Cluster Ready, and of course Perceus and Warewulf makes
Caos NSA an optimal base for all HPC needs.

Caos NSA not only is bundled with Perceus pre-integrated but also has
custom configuration tools (via "sidekick" the Caos configuration toolkit).
While, Perceus is distribution neutral, Caos NSA makes a very smart choice
for doing any type of clustering.

\subsection{Intel Cluster Ready}
The Intel Cluster Ready (ICR) certification is an industry standard
ensuring that the hardware, operating system and software (libraries and
applications) will all work well together. Caos NSA with Perceus is the only
freely available community oriented platform which is ICR certified by Intel
themselves.

\subsection{Infiscale.com}
Both Caos NSA and Perceus is developed and provided to the world under the
GNU GPL by Infiscale.com. Infiscale.com has a multitude of product
enhancements and service offerings based around Caos and Perceus, and we
encourage you to help support continued development and enhancement of
our efforts by supporting Infiscale.com!

\subsection{Intel Rapid Boot}
Intel and Infiscale has created a unique boot and provisioning system which
allows for fast, scalable and large scale provisioning for clustering and
enterprise computing. Using this solution, Perceus is actually embedded on
the motherboard's BIOS chip alongside the Rapid Boot BIOS firmware. This
removes the need to PXE boot the Perceus bootstrap software as it is already
included on the motherboard.

\section{Creating Caos NSA 1.0 installation media}
At the time of this writing the installation version used was 0.9-34, but
you should download the latest CDROM image from:

{\ttfamily
\indent \url{http://mirror.caoslinux.org/Caos-NSA-1.0/install}
}

\noindent {\bf Note: You will want to download either the Cluster or Full
installation media depending on your needs. Other install builds will work
but may require a bit more setup and installation of specific packages.}

Burn this to a CDROM using the utility of your choice (the command to do
this in NSA is {\bf cdrecord} and then put the installation media into the
CDROM drive of the system you wish to install on.

\section{Hardware Setup}
There are lots of models for building clusters, but for the sake of
simplicity we will assume a someone traditional model of a single master
(head) node which has at least two network interfaces. One connected to the
public network of which people will use to gain access to the cluster, and
the second connected via a private network directly to the slave nodes.

From this point on, we will assume that the public network resides on "eth0"
and the cluster network can be anything else (including InfiniBand "ib0").

If installing in VMWare, configure the primary Ethernet device to be either
Bridged or NAT, and then add a second second Ethernet device to be on a custom
virtual network (for example, VMnet2). Configure the CDROM to use the image
file that you downloaded above. When creating the node VM's, they just need
a single Ethernet device configured to also use VMnet2 and remove the following
devices as they are not needed on nodes: sound, usb, cdrom and optionally hard
disk.

\section{Installing Caos NSA}
The installation media that you created from the above step is boot-able
so just insert that into the CDROM drive of the system you wish to install
to, and boot the system. (If the CDROM does not boot, you may need to change
the boot device priority in the BIOS)

Once the system has booted on the installation media, you will be presented
with the installation command line with a brief description of its usage
presented. Install the system with the following command:

\begin{verbatim}
install hostname=headnode.cluster.yourdomain.com
\end{verbatim}

If no networking options are defined, the default action is to configure the
first Ethernet device (eth0) with DHCP. If you wish to make this static, or
even use a different device (instead of eth0), add the following options to
the above command:

\begin{verbatim}
ipaddr=192.168.1.20 netmask=255.255.255.0 gw=192.168.1.1 ifdev=eth1
\end{verbatim}

This will install the system to the primary hard disk {\bf non-interactively}.
Once it is finished, you will be prompted to reboot onto your installed
system. At this point, type the {\tt [ENTER]} key.

\section{System Configuration}
During the first boot, you will be prompted to configure the system via
Sidekick. Use your tab button to navigate to the different fields. This
will guide you through the various required configuration steps.

Navigation through sidekick is done with the arrow keys and {\tt [TAB]}
button. Take a moment to make sure that the right section is highlighted
before pressing the {\tt [ENTER]} key because the {\tt [ENTER]} key will
immediately take all current values and jump to the next page.

Please note that some of the steps maybe slightly different depending on the
specific version of the installer that you are using and the hardware
configuration. Sidekick is designed to be intuitive, so any unexpected steps
should be self explanatory.

\begin{description}

\item[System Information:] Set whatever system information you want
to enter. Confirm the Hostname, and set any of the other fields using the
arrow key to navigate between fields. Press the {\tt [TAB]} button to
select {\tt <OK>} and press {\tt [ENTER]}.

\item[Set Password:] This is the system's "root" user (administrator
password). Type in the system password and press {\tt [ENTER]}. It will
ask you to repeat the password again to confirm and then press
{\tt [ENTER]}.

\item[Timezone:] Set the system's default timezone here by navigating to
the appropriate country or continent and press {\tt [ENTER]}. Then
navigate to the nearest location or timezone to you and press {\tt [ENTER]}
again to set the system's timezone.

\item[Network Device Configuration:] Use this menu to make any changes
to the network configuration for any of the installed devices. If you have
any InfiniBand devices, they will also show up here for configuration of
IPoIB. You will notice that one device is already configured. This device
was configured automatically by the installer.

You will need to configure at least one additional network which is connected
directly and privately to the nodes in the cluster. Use the arrow keys to
select the device you want to configure and press {\tt [ENTER]}.

For most examples, you will be using a Ethernet device (e.g. "eth1"), but if
you are provisioning nodes running Embedded Perceus over InfiniBand then you
will want to select the InfiniBand device connected to the nodes (e.g. "ib0").

\item[Boot Protocol Configuration:] Select static to configure the
device's IP and netmask and press {\tt [ENTER]}.

\item[Static IP Network Configuration:] Input '10.99.99.1' for the IP
address and then use the down arrow to set '255.255.255.0' for the netmask
(do not enter a Gateway). Once configured press {\tt [ENTER]}.

{\tt Note: You may use the above addressing unless it conflicts with one of
the other networks directly attached to this system.}

\item[Network Device Configuration:] You are now back at the main network
device selector. Navigate to the {\tt <Done>} button using the {\tt [TAB]}
key and then press {\tt [ENTER]}.

\item[Firewall Configuration:] If you want to enable the firewall select any
devices that are connected to a public network by using the {\tt [SPACE]} key
then press {\tt [ENTER]}. If you choose to enable the firewall, this will take
you through several firewall related pages where the defaults can be used.

\item[YUM Proxy configuration:] Configure a web proxy to use when installing
or updating packages. If you don't require one, then leave it blank. press
{\tt [ENTER]} when done.

\item[System Profile:] Here you will need to select the "Clustering" and
"ICR" profile to have Sidekick automatically configure your system for
Clustering and Intel Cluster Ready support (You can select or deselect other
profiles as well). When done, press {\tt [ENTER]}.

\item[Enable Perceus Confirmation:] Enable Perceus on your system by selecting
{\tt <Yes>} and press {\tt [ENTER]}.

\item[Perceus Network Device:] Here you should select the device that
is connected to the nodes which we configured earlier. Only devices that have
static IP addresses will show up in this list.

\item[Perceus Configuration:] This screen will allow you to configure
various Perceus options. If all earlier configuration options were successful
then the defaults pre-chosen will work. Make any desired configuration
changes and press {\tt [ENTER]}.

After you pressed {\tt [ENTER]}, Perceus will be automatically configured for
you and will prompt for registration information. The registration should be
entered by you (the end user) and the Perceus developers ask that it be filled
out accurately as it is used for statistical gathering and internal metrics.

\item[OFED InfiniBand Subnet Manager:] If you have an InfiniBand device
installed, then you will be prompted to enable the subnet manager. At least
one subnet manager is required for any InfiniBand communication. If you wish
to enable the subnet manager on this system select {\tt <Yes>} and press
{\tt [ENTER]}.

\item[NAT:] Network Address Translation (NAT) enables other systems to
use this system as their "gateway" to other networks (e.g. the Internet).
Enable this option by selecting {\tt <Yes>} and press {\tt [ENTER]}.

\item[Select network interface for NAT:] Select the network interface
that is connected to the public network and press {\tt [ENTER]}.

\item[Enable Slurm Confirmation:] Slurm is the resource manager and batch
scheduler that is bundled with Caos NSA. It allows one to submit batch jobs
(shell scripts) to the scheduler and the scheduler will run them in order
(FIFO) on the first available nodes. Select {\tt <Yes>} and press
{\tt [ENTER]}.

\item[System Update:] This will do an automatic check for system
updates. If you wish to do this select {\tt <Yes>} and press {\tt [ENTER]}.

\item[Register Your Installation:] The Caos NSA developers request
that they are notified of successful system installations. You are not
required to do this, and it is only used for them to keep track of
installation numbers. If you don't mind doing this select {\tt <Yes>} and
press {\tt [ENTER]}.

\item[All Done!] System configuration is complete.... Press
{\tt [ENTER]} to continue to the login prompt.

\end{description}

At this point the system will continue booting and end up at the login
screen. Go ahead and log in as "root" with the password you configured
earlier.

\section{Perceus Configuration}

\subsection{Booting Nodes}
Perceus is already running and is ready to start to booting nodes. Go ahead
and boot a test node at this point. If the node and network is configured
properly, it will PXE boot or boot via the embedded Perceus image and will
stop at:

\begin{verbatim}
ERROR: No configured VNFS image for n0000!
\end{verbatim}

At this point we have not imported any Virtual Node File System's into
Perceus, thus this error message is expected and demonstrates that Perceus
has gained control of the node and we can provision once configured.

\subsection{Perceus Modules}
Perceus modules allow one to inject additional functionality into the
various stages of the provisioning process. The "masterauth" module will
synchronize the user accounts from the master node to all slaves booting
and/or in the ready state.

\begin{verbatim}
# perceus module activate masterauth
Perceus module 'masterauth' has been set active in 'init/all'
Perceus module 'masterauth' has been set active in 'ready/all'
# perceus module states
init/all:
   masterauth
ready/all:
   masterauth
\end{verbatim}

Activating the perceus module without a specific state argument will cause it
to be activated in its default state(s).

If you are running Perceus embedded on the node motherboard, then you will
need to obtain the initbins-x86 Perceus module from Infiscale.com, download,
import and activate it with Perceus.

\begin{verbatim}
# perceus module import /root/initbins-x86-1.4.pmod
initbins-x86-1.4 has been imported
# perceus module activate initbins-x86-1.4
Perceus module 'initbins-x86-1.4' has been set active in 'init/all'
# perceus module states
init/all:
   initbins-x86-1.4
   masterauth
ready/all:
   masterauth
\end{verbatim}


\subsection{VNFS Capsules}
Perceus handles the provisioning of nodes via the Virtual Node File
System (VNFS) and Perceus has a mechanism of encapsulating them into a
single file commonly referred to as the "VNFS capsule". You will need to
import the VNFS capsule that you want to run on the nodes.

The Caos NSA development team has not only installers for Caos NSA available
but also pre-made VNFS capsules on their download sites and mirrors. You
can download one from {\tt http://mirror.caoslinux.org/Caos-NSA-1.0/vnfs/}.
At the time of this writing, there are two primary builds available:
caos-nsa-node and caos-nsa-node-icr. The 'icr' version is the Intel Cluster
Ready build. It is bigger as it contains a much larger compatibility base.
Once downloaded, you can perform the following command:

\begin{verbatim}
# perceus vnfs import caos-nsa-node-icr-0.9-34-1.stateless.x86_64.vnfs
\end{verbatim}

This command will take a moment to operate and will prompt you for some
basic information. Answer the questions appropriately for your needs (most
defaults should be used).

\subsection{Node configuration}
At this point, we have booted at least one node(s), and we have a VNFS
imported into Perceus but the node has not been set to any VNFS so it is
still showing the ERROR on the console.

Perceus uses attributes for each node which can be set with the Perceus node
"set" command. Use the set command to assign the above-mentioned VNFS capsule
to the booted node:

\begin{verbatim}
# perceus node set vnfs caos-nsa-node-icr-0.9-34-1.stateless.x86_64 n0000
\end{verbatim}

Once that command has been run and confirmed, the node will be automatically
provisioned (no services to restart, or reboot necessary). Additional nodes
booted and added to the Perceus database will also need to be set to use a
particular VNFS capsule. You can configure default actions in the
{\tt defaults.conf} file described below in the configuration file section.

The "perceus node" and "perceus group" commands can take multiple arguments and
the arguments can be glob's, node ranges, etc.. The following example will set
the node group to "newgroup" on the listed nodes.

\begin{verbatim}
# perceus node set group newgroup n0000 n00[01-20] n01*
\end{verbatim}

\subsection{Node Information}
To view node information, you can use the following commands:

\begin{verbatim}
# perceus node summary
# perceus node stats
# perceus node show
\end{verbatim}

All of the above commands can take node arguments in the form of a list, glob,
or ranges as the above "set" command does.

\subsection{Using Node Groups}
Groups offer a way of combining multiple nodes and operating on them in unison.
For example, using the above example where we set some nodes to a group called
"newgroup", we can demonstrate the following:

\begin{verbatim}
# perceus group set enabled 0 newgroup
# perceus group set vnfs rhel-5.1-1.stateless.x86_64 newgroup
# perceus group set group rhel5 newgroup
\end{verbatim}

\subsection{Modifying the VNFS}
You may need to make changes to the default VNFS capsule. To do this, you
should mount the VNFS image (using a Perceus mount), make changes and then
unmount the image. Once the VNFS is mounted, you can operate on it like a
standard directory or chroot file system.

\begin{verbatim}
# perceus vnfs mount caos-nsa-node-icr-0.9-34-1.stateless.x86_64
Mounting VNFS 'caos-nsa-node-icr-0.9-34-1.stateless.x86_64'...
The VNFS can be found at: /mnt/caos-nsa-node-icr-0.9-34-1.stateless.x86_64
# yum --installroot /mnt/caos-nsa-node-icr-0.9-34-1.stateless.x86_64 install ntp
   ...
# chroot /mnt/caos-nsa-node-icr-0.9-34-1.stateless.x86_64 /sbin/chkconfig ntp on
# perceus vnfs umount caos-nsa-node-icr-0.9-34-1.stateless.x86_64
Un-mounting VNFS 'caos-nsa-node-icr-0.9-34-1.stateless.x86_64'...
This will take some time as the image is updated and compressed...
\end{verbatim}


\subsection{Configuration Structure}
The default location for configuration files is {\tt /etc/perceus} which is
where you will find the following files.

\begin{description}
\item[perceus.conf:] This configuration specifies some of the primary Perceus
configuration options. Any changes to this file will require a restart of the
Perceus init script ({\tt /sbin/service perceus restart}. Here you can
reconfigure Perceus to use HTTPD instead of NFS to do the provisioning as well
as set the NFS or HTTPD server to another box (e.g. a scalable and redundant
server).

\item[defaults.conf:] Configurations here set how new nodes that are added to
the Perceus database are handled. Everything from default node name format,
group and VNFS can be automatically set when a new node is configured.

\item[modules:] Any modules that are loaded should (but is not forced) to use
this location for configuration files.

\end{description}

\subsection{Integration with Warewulf 2.9}
As Perceus replaces the provisioning functionality of Warewulf <= 2,
Warewulf 3 targets a different niche in HPC cluster computing. At the
moment, the freely available Warewulf 3 development comes with NSA and has
been integrated with Perceus as it stands now. You can see this by typing
the command {\tt wwtop} to see the status of the running nodes.

\subsection{A Working Perceus Cluster}
Congratulations, you now have a working Perceus cluster! This cluster can be
used as the starting point for a High Performance Computing solution, web
server farm, large infrastructure deployment and management utilities, disk IO,
CPU farm, computer workstation lab, virtualization host framework or anything
else you can think of!

Caos NSA and Perceus lays the perfect architecture for turn key systems or
provides the base environment for minimal or extreme customization and
appliances.

\section{Commercial Features and Services}

Perceus is licensed free of charge via the GNU GPL.  This means that all costs
incurred by its development, testing, and distribution have been borne by
Infiscale.com.  We encourage both corporate partners and licensees to contact
us as it helps us to maintain our freely available product.

If you wish to distribute Perceus on a commercial basis (including utilizing
it with commercial software or with a hardware purchase) you should pursue a
commercial relationship with Infiscale.com.

Don't forget to check with Infiscale to see how they can help you as a user too!
Supporting Infiscale, supports freely available open source software like this.

\indent \end{document}

